\documentclass[french]{article}

\usepackage[T1]{fontenc}
\usepackage[french]{babel}
\usepackage{geometry}

\usepackage{pythonhighlight}
\definecolor{keywordcolour}{HTML}{5e81ac}
\definecolor{literatecolour}{HTML}{5e81ac}
\definecolor{stringcolour}{HTML}{8fbcbb}

\usepackage{listings}
\lstset{
    basicstyle=\small\ttfamily,
    columns=flexible,
    breaklines=true
}

\title{De la quantique en cryptographie}
\author{Élie Besnard, Malo Leroy, \\
Yun Marcola--da-Cunha Macedo}

\begin{document}

\maketitle

\tableofcontents

\section{Ordinateur quantique}

\subsection{Qubits}
\inputpython{Simulations/Ordinateur/qubit.py}{0}{190}

\subsection{Portes et oracles}
\inputpython{Simulations/Ordinateur/portes.py}{0}{95}
\inputpython{Simulations/Ordinateur/oracle.py}{0}{105}

\subsection{Calcul formel}
\inputpython{Simulations/Ordinateur/calcul.py}{0}{986}

\subsection{Fonctions utiles}
\inputpython{Simulations/Ordinateur/fonctions_utiles.py}{0}{53}


\section{Ordinateur quantique : tests}

\subsection{Qubits}
\inputpython{Simulations/Ordinateur/qubit_test.py}{0}{190}

\subsection{Portes et oracles}
\inputpython{Simulations/Ordinateur/portes_test.py}{0}{95}
\inputpython{Simulations/Ordinateur/oracle_test.py}{0}{69}

\subsection{Calcul formel}
\inputpython{Simulations/Ordinateur/calcul_test.py}{0}{986}

\subsection{Fonctions utiles}
\inputpython{Simulations/Ordinateur/fonctions_utiles_test.py}{0}{53}

\subsection{Analyses d'efficacité}
\inputpython{Simulations/Ordinateur/performance.py}{0}{140}


\section{Algorithmes}

\subsection{Deutsch-Jozsa et Bernstein-Vazirani}
\inputpython{Simulations/Ordinateur/dj.py}{0}{27}

\subsection{Grover}
\inputpython{Simulations/Ordinateur/grover.py}{0}{52}

\subsection{Shor}
\inputpython{Simulations/Ordinateur/shor.py}{0}{103}

\subsection{Parité}
\inputpython{Simulations/Ordinateur/pair_impair.py}{0}{27}
\inputpython{Simulations/Ordinateur/parite.py}{0}{22}

\subsection{Tests d'algorithmes}
\inputpython{Simulations/Ordinateur/algorithmes_test.py}{0}{103}


\section{Polarisation}
\inputpython{Simulations/Intrication/animation mesure intrication.py}{0}{74}

\section{Interface graphique}

\lstinputlisting{Interface/index.html}
\lstinputlisting{Interface/script.js}
\lstinputlisting{Interface/stylesheet.css}

\end{document}
