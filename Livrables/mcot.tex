\documentclass{article}
\usepackage[utf8]{inputenc}
\usepackage{geometry}

\title{\textbf{\underline{Titre et Mise en Cohérence des Objectifs du TIPE}}}
\author{}
\date{}
\geometry{hmargin=2.5cm,vmargin=1.5cm}

% j'ai bricolé ça rapidement pour avoir les grandes idées, si tu veux changer la mise en page ou autre chose, hésites pas. J'ai fait ça comme je fait d'habitude mais change ce que tu veux

\begin{document}

\maketitle

\paragraph{Groupe :} Élie Besnard, Malo Leroy, Yun Marcola

\paragraph{Ancrage au thème :} L’augmentation de la connectivité dans les zones urbaines et de la puissance de calcul à disposition entraîne le besoin de développer des transmissions de données toujours plus sécurisées. \newline ( 28 mots ) % pas terrible mais l'idée est là

\paragraph{Motivation :} Nous avons choisi ce sujet afin de nous pencher sur la vaste théorie de la Mécanique Quantique afin d'en décourvrir certains principes et d'en appréhender une application concrète.

\paragraph{Mots-Clés :} 

\begin{itemize}
    \item Informatique / Computer Science
    \item Cryptographie / Cryptography
    \item Mécanique Quantique / Quantum Mechanics
    \item Algèbre Linéaire / Linear Algebra
    \item Arithmétique / Arithmetic
\end{itemize}

\paragraph{Positionnement Thématique :} 

\begin{itemize}
    \item Informatique ( Cryptographie )
    \item Physique ( Mécanique )
    \item Mathématiques ( Algèbre Linéaire )
\end{itemize} 

\paragraph{Bibliographie :}

\begin{enumerate}
    \item \underline{Mécanique Quantique} de Claude Cohen-Tanoudji, Bernard Diu et Franck Laloë, éditions Hermann
    \item \underline{Le Cours de Physique de Feynman}, tome 5, de Richard Feynman, éditions Dunod
    \item https://fr.wikipedia.org/wiki/Algorithme\_de\_Shor Principe de l'algorithme de Shor
\end{enumerate}

\paragraph{Bibliographie Commentée :} En premier lieu, nous nous sommes intéressés au fondement de la Cryptographie Quantique : l'algorithme de Shor $(3)$. Cet algorithme, destiné à être exécuté sur un ordinateur quantique permet de factoriser un entier en produit de facteurs premiers avec une remarquable efficacité. Mais pour pouvoir approfondir nos recherches sur le sujet, nous avons du nous familiariser avec les bases du formalisme de la mécanique quantique et de son cadre mathématique. Nous avons pour cela parcouru différents ouvrages $(1)$ et $(2)$ de grandes figures de la Physique Quantique.

\paragraph{Problématique :} Comment la Mécanique Quantique pourrait nous permettre de sécuriser nos échanges de données pour les années à venir ?

\end{document}
