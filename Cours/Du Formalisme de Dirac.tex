\documentclass{article}
\usepackage[utf8]{inputenc}

\title{\textbf{\underline{Du Formalisme de Dirac}}}
\usepackage{geometry}
\usepackage{amssymb}
\usepackage{braket}
\geometry{hmargin=2.5cm,vmargin=1.5cm}
\author{}
\date{}

\begin{document}

\maketitle

\section{Strucutre de $\mathfrak{F}$ :}

On désigne par $\mathfrak{F}$ l'ensemble des fonctions d'onde de carré sommable. Une fonction est dite de carré sommable si $\displaystyle \int_{-\infty}^{+\infty}f(t)dt$ existe et est fini.
$\mathfrak{F}$ est un espace vectoriel.
$\newline \newline$
Soient $\varphi ,\psi \in \mathfrak{F}$, on définit le produit scalaire de la façon suivante : $(\varphi, \psi) = \displaystyle \int d^3r \varphi^*\psi$ Le produit scalaire sur $\mathfrak{F}$ est antilinéaire par rapport à le deuxième variable, définie positive.

\section{Opérateurs Linéaires :}

Un opérateur linéaire est un objet qui, à toute fonction d'onde $\psi \in \mathfrak{F}$, associe une autre fonction $\psi' \in \mathfrak{F}$. Soit $A$ un opérateur linéaire, on note $A\psi(r) = \psi'(r)$.
Un opérateur linéaire satisfait les propriétés de linéarités.
$\newline$
\underline{Exemples :} l'opérateur parité $\Pi\psi(x,y,z) = \psi(-x,-y,-z)$, l'operateur dérivée $D_x\psi(x,y,z) = \displaystyle \frac{\partial\psi}{\partial x}$
$\\$ $\\$
On définit le produit d'opérateurs de la façon suivante : $(AB)\psi(r) = A[B\psi(r)]$

\section{Base Orthonormée :}

Soit $\{u_i(r) \in \mathfrak{F}\}$ une base de $\mathfrak{F}$. Cette base est dite orthonormée si $(u_i,u_j) = \delta_{i,j}$ ou $\delta$ désigne le symbole de Kroenecker.

\section{Espace des états $\mathcal{E}$ :}

On définit $\mathcal{E}$ l'espce des états d'un système en associant un vecteur $\ket{\psi} \in \mathcal{E}$ à toute fonction d'onde $\psi(r)$.
\underline{Attention :} $\ket{\psi}$ ne dépend plus de $r$.

\section{Espace dual $\mathcal{E^*}$ :}

Une fonctionnelle linéaire est une opération linéaire $\chi$ linéaire qui à tout vecteur $\ket{\psi} \in \mathcal{E}$ associe un nombre complexe. L'ensemble des fonctionnelles linéaires est un espace vectoriel, appelé espace dual et noté $\mathcal{E^*}$. Les éléments de $\mathcal{E^*}$ sont notés $\bra{\chi}$. Soit $\ket{\psi} \in \mathcal{E}$ et $\bra{\varphi} \in \mathcal{E^*}$, $\varphi(\ket{\psi}) = \braket{\varphi|\psi}$. Dans $\mathcal{E}$, le produit scalaire est une forme antilinéaire définie positive.

\section{Notation $\ket{\psi}\bra{\varphi} :$}

Soient $\ket{\psi} \in \mathcal{E}$ et $\bra{\varphi} \in \mathcal{E^*}$. Considérons de plus $\ket{\chi} \in \mathcal{E}$. Alors $\ket{\psi}\braket{\varphi|\chi} \in \mathcal{E}$. L'objet $\ket{\psi}\bra{\varphi}$ associe donc un vecteur de $\mathcal{E}$ à un autre vecteur de $\mathcal{E}$. Cet objet est donc un opérateur linéaire.

\end{document}
 