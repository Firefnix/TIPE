\documentclass[french]{article}
\usepackage{geometry}
\geometry{hmargin=2.5cm,vmargin=1.5cm}
\usepackage[utf8]{inputenc}

\title{\textbf{\underline{TP 1 : Loi de Malus}}}
\author{}

\begin{document}

\maketitle

\section{Mise en situtation}

Afin de pouvoir mesurer expérimentalement la polarisation d'une onde électromagnétique, il est nécessaire de trouver un lien entre une grandeur mesurable et cette polarisation, n'étant pas directement mesurable dans le cas d'une onde électro-magnétique. Pour cela, nous allons étudier la loi de Malus, reliant l'angle de polarisation $\theta$, l'intensité lumineuse incidente $I_0$ et sortante $I$ par la relation :

$$\fbox{I = I_0\cos^2(\theta)}$$

\section{Détails du TP}

\subsection{Objectif du TP}

Valider expérimentalent la loi de Malus en comparant résultats théoriques et mesures expérimentales.

\subsection{Matériel}

\begin{itemize}
    \item un laser
    \item une cellule photosensible
    \item un banc optique
    \item un filtre polarisant
\end{itemize}

\section{Déroulement}

\paragraph{Mesure de $I_0$ :} Il s'agit tout d'abord de mesurer l'intensité incindente, inhérente à la source utilisée. Pour cela, on ne place pas de filtre polariseur et on mesure simplement l'intensité de la lumière émise par la source.

\paragraph{Mesure de $(\theta, I)$ :} Pour différentes valeurs de $\theta$, on mesure l'intensité sortante $I$ afin de comparer ces résultats avec le modèle théorique.

\paragraph{Comparaison :} On compare maintenant les résultats obtenus avec les valeurs prédites par le modèle théorique afin d'en étudier la validité.

\section{Résultats}

\section{Conclusion}

\end{document}
