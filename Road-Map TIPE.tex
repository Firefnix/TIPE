\documentclass{article}
\usepackage{geometry}
\geometry{hmargin=2.5cm,vmargin=1.5cm}
\usepackage[utf8]{inputenc}

\title{\textbf{\underline{Road-Map TIPE}}}
\author{}
\date{}

\begin{document}

\maketitle

\section*{\underline{Mise en Situation :}}
Aujourd'hui, l'immensité du volume de données échangées et l'ampleur des enjeux qui y sont liés (sécurité bancaire, confidentialité, respect de la vie privée...) ainsi que la puissance de calcul croissante  dont nous disposons nous obligent à mettre en oeuvre des moyens de sécuriser nos échanges de données toujours plus sophistiqués. L'avènement de l'ère de l'Ordinateur Quantique pourrait mettre en péril tous les moyens actuellement mis en oeuvre. Toutefois, de nouvelles méthodes faisant appel à la mécanique quantique pourraient nous permettre d'atteindre un niveau de sécurité proche de la perfection, au prix de forte contraintes techniques de mise en place.

\section{Préambule :}

\begin{itemize}
    \item Rapide présentation de la Cryptographie
    \item Notion de Protocole
    \item " Cassabilité " d'un Protocole
\end{itemize}

\section{L'Ordinateur Quantique : Arme Nucléaire de la Cryptographie}

\paragraph{L'Algorithme de Shor :} Fondement de la problématique $\rightarrow$ possibilité de casser tous les protocoles Classiques actuels.

\begin{itemize}
    \item Implémentation 
    \item Mise en évidence de la nécessité de l'Ordinateur Quantique
    \item Présentation de données statistiques ( matplotlib )
    \item Animations ? ( haute valeur ajoutée mais reste à voir ce qui serait intéressant à faire )
\end{itemize}
    
\section{La Révolution de la Cryptographie Quantique :}

\begin{itemize}
    \item Rapide présentation des fondements de la Physique Quantique $\rightarrow$ inégalités d'Heisenberg, fonction d'onde, modèle probabiliste...
    \item Le Protocole E91 $\rightarrow$ principe de fonctionnement
    \item Le Phénomène d'Intrication Quantique
    \begin{itemize}
        \item Vulgarisation du Principe
        \item Etude Théorique du Phénomène
    \end{itemize}
\end{itemize}

\end{document}
